\documentclass{beamer}

\begin{document}

\title{Introduction to Quantum Espresso}   
\author{Otto Kohul\'{a}k \newline kohulak@fmph.uniba.sk} 
\date{\today} 

\frame{\titlepage} 

\begin{frame}
  \frametitle{What is QE}
  \begin{itemize}
    \item Quantum ESPRESSO (opEn-Source Package for Research in Electronic Structure, Simulation, and Optimization)
    \item GPL License
    \item Written in Fortran and C
    \item Supports mpi parallelization
    \item It is capable of:
    \begin{itemize}
      \item Ground state calculations
      \item Structural Optimizations
      \item Ab-initio MD (BO and CP)
      \item Reponse properties
      \item Spectroscopic properties
    \end{itemize}
  \end{itemize}
\end{frame}

\begin{frame}
  \frametitle{Links}
  \begin{itemize}
    \item http://www.quantum-espresso.org/
    \item http://www.quantum-espresso.org/tutorials/
    \item https://github.com/addman2/KvantSim
  \end{itemize}
\end{frame}

\begin{frame}
  \frametitle{Message Passing Interface (MPI)}
  \begin{itemize}
    \item Is na interface for parallel computing, based od sending a messages between computing nodes
    \item Most common implementations:
    \begin{itemize}
      \item OpenMPI
      \item MPICH
      \item intel MPI (impi)
    \end{itemize}
    \item Basic support for FORTRAN and C/C++
  \end{itemize}
\end{frame}

\begin{frame}
  \frametitle{How to compute}
  \begin{itemize}
    \item Log via ssh to 158.195.19.X (X are last 3 numbers of lecture room - 259)
    \item Run: python /share/import\_environment.py (and follow instructions)
    \item Use settool tool to import desired applications
    \begin{itemize}
      \item settool mpirun
      \item settool espresso
    \end{itemize}
  \end{itemize}
\end{frame}

\end{document}

